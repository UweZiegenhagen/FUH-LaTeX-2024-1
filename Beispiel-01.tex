\documentclass[12pt,ngerman,parskip=full]{scrartcl}

\usepackage[T1]{fontenc}
\usepackage{booktabs}
\usepackage{babel}
\usepackage{graphicx}
\usepackage{csquotes}
\usepackage{paralist}
\usepackage{xcolor}
\usepackage{microtype}
\usepackage{palatino}

\newcounter{abschnitt}
\setcounter{abschnitt}{1}

\newcommand{\abschnitt}{\subsection*{\theabschnitt}\stepcounter{abschnitt}}

\begin{document}

\abschnitt

Sage zu dir in der Morgenstunde: Heute werde ich mit einem unbedachtsamen, undankbaren, unverschämten, betrügerischen, neidischen, ungeselligen Menschen zusammentreffen. Alle diese Fehler sind Folgen ihrer Unwissenheit hinsichtlich des Guten und des Bösen.Es war ein stoischer Grundsatz, dessen Ursprung auf Zeno zurückgeführt wurde, daß die meisten Menschen nur aus Dummheit böse sind. Ich aber habe klar erkannt, daß das Gute seinem Wesen nach schön und das Böse häßlich ist,Diesen Satz hatte Zeno aufgestellt, aber dieselbe Lehre findet sich schon bei Plato. daß der Mensch, der gegen mich fehlt, in Wirklichkeit mir verwandt ist, nicht weil wir von demselben Blut, derselben Abkunft wären, sondern wir haben gleichen Anteil an der Vernunft, der göttlichen Bestimmung. Keiner kann mir Schaden zufügen, denn ich lasse mich nicht zu einem Laster verführen. Ebensowenig kann ich dem, der mir verwandt ist, zürnen oder ihn hassen; denn wir sind zur gemeinschaftlichen Wirksamkeit geschaffen, wie die Füße, die Hände, die Augenlider, wie die obere und untere Kinnlade.Derartige Vergleiche waren bei den Alten nichts Seltenes. Darum ist die Feindschaft der Menschen untereinander wider die Natur; Unwillen aber und Abscheu in sich fühlen ist eine Feindseligkeit.

\abschnitt

Was ich auch immer sein mag, es ist doch nur ein wenig Fleisch, ein schwacher Lebenshauch und die leitende Vernunft. Laß die Bücher,Antonin war durch seine Regentenpflichten so sehr beschäftigt, daß er seine Neigung zum Lesen unterdrücken mußte. (Bücher konnten damals nur die Reichen kaufen, die philosophischen Schriften wurden oft mit mehreren Talenten, d. h. mehreren Tausend Talern bezahlt.) die Zerstreuung, es fehlt dir die Zeit. Betrachte dich als einen, der im Begriff ist zu sterben, verachte dieses Fleisch: Blut, Knochen, ein zerbrechliches Gewebe, aus Nerven, Puls- und Blutadern zusammengeflochten. Betrachte diesen Lebenshauch selbst; was ist er? Nur Wind, und nicht einmal immer derselbe, sondern jeden Augenblick ausgeatmet und wieder eingeatmet. Das Dritte ist die gebietende Vernunft. Auf folgendes mußt du bedacht sein: Du bist alt; gib nicht mehr zu, daß sie eine Sklavin sei, daß sie durch einen wilden Trieb dahingerissen werde oder gegen das jetzige Geschick murre oder durch das künftige erschüttert werde.

\abschnitt

Alles ist voll von Spuren göttlicher Vorsehung. Auch die zufälligen Ereignisse sind nichts Unnatürliches, sind abhängig von dem Zusammenwirken und der Verkettung der von der Vorsehung gelenkten Ursachen. Alles geht von der Vorsehung aus. Hiermit verknüpft sich sowohl die Notwendigkeit als auch das, was zur Harmonie des Weltganzen nützlich ist, wovon du ein Teil bist. Was mit dem großen Ganzen übereinstimmt und was zur Erhaltung des Weltplanes dient, das ist für jeden Teil der Natur gut. Die Harmonie der Welt wird erhalten sowohl durch die Veränderungen der Grundstoffe als auch der daraus bestehenden Körper. Das genüge dir, das möge dir stets zur Lehre dienen. Den BücherdurstMehrere Stoiker waren gegen das viele Bücherlesen. Seneka sagt, daß sich selbst zu studieren den Vorzug verdiene vor dem Studium vieler Bücher. vertreibe, damit du nicht murrend sterbest, sondern mit wahrem Seelenfrieden und dankbarem Herzen gegen die Götter.

\abschnitt

Erinnere dich, seit wie lange du die Ausführung verschiebst und wie oft dir die Götter günstige Gelegenheit gegeben haben, die du unbenutzt gelassen. Du solltest es doch einmal empfinden, von welcher Welt du ein Teil bist und von welchem Herrn der Welt dein Dasein seinen Ursprung hat, daß die Zeit für dich schon abgegrenzt ist; und wenn du sie nicht auf die Seelenheiterkeit verwendest, so schwindet sie dahin, und du schwindest selbst dahin, und sie kehrt nie zurück.

\abschnitt

Denke zu jeder Tageszeit daran, in deinen Handlungen einen festen Charakter zu zeigen, wie er einem Römer und einem Mann geziemt, einen ungekünstelten, sich nie verleugnenden Ernst, ein Herz voll Freiheits- und Gerechtigkeitsliebe. Verscheuche jeden anderen Gedanken, und das wirst du können, wenn du jede deiner Handlungen als die letzte deines Lebens betrachtest, frei von Überstürzung, ohne irgendeine Leidenschaft, die der Vernunft ihre Herrschaft entzieht, ohne Heuchelei, ohne Eigenliebe und mit Ergebung in den Willen des Schicksals. Du siehst, wie wenig zu beobachten ist, um ein friedliches, von den Göttern beglücktes Leben zu führen. Die Befolgung dieser Lehre ist ja alles, was die Götter von uns verlangen.

\abschnitt

Schmähe dich, ja schmähe dich, Seele! Dich zu ehren, wirst du keine Zeit mehr haben. Unser Leben ist flüchtig, das deinige ist fast schon am Ziele, und du hast keine Achtung vor dir, denn du suchst deine Glückseligkeit in den Seelen anderer.Nach den Lehren der Stoiker soll der Mensch nach einem naturgemäßen Leben trachten und das Urteil anderer verachten.

\abschnitt

Warum dich durch die Außendinge zerstreuen? Nimm dir Zeit, etwas Gutes zu lernen und höre auf, dich wie im Wirbelwind umhertreiben zu lassen. Hüte dich noch vor einer andern Verirrung, denn es ist auch Torheit, sich das Leben durch zwecklose Handlungen schwer zu machen;Marc Aurel sagt, daß die Seele des Menschen sich mit Schmach bedeckt (6), wenn sie bei ihrer Handlung kein Ziel verfolgt, sondern ihr Tun dem Zufall überläßt. man muß ein Ziel haben, auf das sich alle unsere Wünsche, alle unsere Gedanken richten.

\abschnitt

Es ist noch nie jemand unglücklich geworden, weil er sich nicht um das, was in der Seele eines andern vorgeht, gekümmert hat; aber diejenigen, die nicht mit Aufmerksamkeit den Bewegungen ihrer eigenen Seele folgen, geraten notwendig ins Unglück.Wer sich immer nur um andere kümmert und nicht um sich, lernt nie sich selbst erkennen.

\abschnitt

Halte dir immer gegenwärtig, welches die Natur des Weltalls und welches die deinige ist, welche Beziehungen diese zu jener hat und welch einen Teil von welchem Ganzen du ausmachst, und dann, daß niemand es dir verwehren kann, dasjenige zu tun oder zu sagen, was mit der Natur, von der du selbst ein Teil bist, übereinstimmt.

\abschnitt

TheophrastTheophrast war Schüler und Nachfolger des Aristoteles. Von seinen Schriften sind nur die Charakterschilderungen und ein botanisches Werk erhalten. sagt bei der Vergleichung der Vergehungen, insofern man nach den gewöhnlichen Begriffen eine solche anstellen mag,Nach der stoischen Lehre waren alle Sünden gleich, weil jede Sünde vernunftwidrig ist. mit Recht, daß die Übertretungen aus Begierden schwerer seien als die aus Zorn. In der Tat entfernt sich der Zornige mit einer gewissen Mißstimmung, mit einem heimlichen Verdruß von der Vernunft; aber derjenige, der aus Begierde sündigt, von der Wollust überwältigt, zeigt sozusagen in seinen Fehlern mehr Unmäßigkeit, mehr unmännliche Schwäche. Es ist daher ein richtiges Wort, würdig der Philosophie, daß aus böser Lust sündigen strafbarer sei, als aus Mißstimmung. Gewiß, der Zürnende stellt sich mehr als ein Mensch dar, dem vorher Unrecht geschah und der durch Schmerz zum Zorn fortgerissen wird; der andere hingegen neigt sich aus freien Stücken zur Ungerechtigkeit, fortgerissen zur Befriedigung seiner Begierden.

\abschnitt

All dein Tun und Denken sei so beschaffen, als solltest du möglicherweise im Augenblick aus diesem Leben scheiden. Aus der Mitte der Menschen zu scheiden, hat nichts Schreckliches, wenn es Götter gibt, denn sie werden dich nicht dem Unglück preisgeben; gibt es hingegen keine Götter oder kümmern sie sich nicht um die menschlichen Angelegenheiten, was liegt dann daran, in einer Welt ohne Götter und ohne Vorsehung zu leben? Doch es gibt Götter, und sie sorgen für die Menschen. Sie haben dem Menschen die Macht gegeben, nicht in die wirklichen Übel zu verfallen. Es gibt kein denkbares Übel, bei dem die Götter nicht vorgesorgt hätten, daß der Mensch die Macht habe, sich davor zu hüten. Wie aber sollte das, was den Menschen selbst nicht unglücklicher macht, des Menschen Leben unglücklicher machen können? Die Allnatur hätte weder unwissentlich noch wissentlich, indem sie nämlich unfähig gewesen wäre, so etwas zu verhüten oder wieder gutzumachen, einer solchen Nachlässigkeit sich schuldig gemacht, und ebensowenig aus Unvermögen oder Ungeschicklichkeit ein so großes Versehen begangen, guten und bösen Menschen Güter und Übel in gleichem Maße ohne Unterschied zukommen zu lassen. Tod und Leben, Ehre und Unehre, Schmerz und Vergnügen, Reichtum und Armut, alle diese Dinge mögen den Bösen wie den Guten ohne Unterschied zuteil werden, denn sie sind an sich weder ehrbar noch schändlich, sind also in Wahrheit weder ein Gut noch ein Übel.

\abschnitt

Wie schnell doch alles verschwindet! In der Welt die Menschen selbst, in der Zeit ihr Andenken! Was ist alles Sinnliche, besonders das, was uns durch Wollust reizt oder durch Schmerz erschreckt, endlich das, was uns durch Scheingröße Rufe der Bewunderung entlockt: wie unbedeutend und verächtlich, wie niedrig, hinfällig und tot! Dies zu erwägen, geziemt dem denkenden Menschen. Wer sind selbst diejenigen, deren Meinungen und Reden Ruhm verleihen? Was ist der Tod? Wenn man ihn für sich allein betrachtet und in Gedanken das davon absondert, was in der Einbildung damit verbunden ist, so wird man darin nichts anderes erblicken als eine Wirkung der Natur. Wer sich aber vor einer Naturwirkung fürchtet, ist ein Kind. Noch mehr, der Tod ist nicht bloß eine Wirkung der Natur, sondern eine für die Natur heilsame Wirkung. Betrachte endlich, wie und durch welchen Teil seines Wesens der Mensch mit Gott in Berührung steht und in welchem Zustande er sich dann befindet, wenn dieses Körperteilchen zerstäubt ist.

\abschnitt

Nichts ist jämmerlicher als ein Mensch, der alles ergründen will, der die Tiefen der Erde, wie jener Dichter sagt,Ein Wort Pindars, dessen sich Plato bedient, um den wahren Philosophen zu kennzeichnen. durchforscht und, was in der Seele seines Nebenmenschen vorgeht, zu erraten sucht, ohne zu bedenken, daß er sich genügen lassen sollte, mit dem Genius, den er in sich hat, zu verkehren und diesem aufrichtig zu dienen. Dieser Dienst aber besteht darin, ihn vor jeder Leidenschaft, Eitelkeit und Unzufriedenheit mit dem Tun der Götter und Menschen zu bewahren. Denn was von den Göttern kommt, verdient unsere Ehrerbietung wegen der Vortrefflichkeit, und was von den Menschen kommt, unsere Liebe wegen der Verwandtschaft, die zwischen uns besteht, manchmal verdient es eine Art Mitleid wegen ihrer Unkenntnis des Guten und Bösen; sie sind wie Blinde oder so, wie wenn jemand Weiß und Schwarz voneinander nicht zu unterscheiden vermag.

\abschnitt

Und wenn du dreitausend Jahre lebtest, selbst dreißigtausend, so erinnere dich dennoch, daß keiner ein anderes Leben verliert als das, was er wirklich lebt, und kein anderes lebt, als das, was er verliert. Das längste Leben kommt also mit dem kürzesten auf eins hinaus. Der gegenwärtige Zeitpunkt ist für alle von gleicher Dauer, welche Ungleichheit es auch in der Dauer des Vergangenen geben mag, und den man verliert, erscheint nur wie ein Augenblick; niemand kann weder die Vergangenheit noch die Zukunft verlieren, denn wie sollte man ihm das rauben können, was er nicht besitzt? Man muß sich also diese beiden Wahrheiten merken, die eine, daß alles sich im ewigen, unveränderlichen Kreislauf befindet, und daß es von keiner Wichtigkeit ist, dieselben Dinge hundert oder zweihundert Jahre oder eine grenzenlose Zeit zu beobachten;Zu Marc Aurels Zeit war es eine ausgemachte Wahrheit, daß es nichts Neues in der Welt gibt, sondern daß alles wiederkehrt. die andere, daß der im höchsten Lebensalter und der sehr jung Sterbende beide das gleiche verlieren. Sie verlieren nur den gegenwärtigen Zeitpunkt, weil sie nur diesen allein besitzen und weil man das, was man nicht besitzt, nicht verlieren kann.

\abschnitt

Alles beruht auf der Meinung. Die Schlußfolgerungen des Zynikers MonimusEin Schüler des Diogenes und Krates, der alle Erkenntnis für bloße Meinung erklärte. sind ganz richtig und gewähren auch Nutzen, wenn man sie auf das, was daran wahr ist, einschränkt.

\abschnitt

Die Seele des Menschen bedeckt sich vornehmlich dann mit Schmach, wenn sie gleichsam eine Geschwulst, ein krankhaftes Geschwür in der Welt wird. Denn über Dinge, die uns begegnen, unzufrieden sein, heißt so viel wie sich von der allgemeinen Natur, die die Natur aller besonderen Wesen in sich faßt, lossagen. Ferner entehrt sie sich durch Abneigung gegen einen Menschen oder wenn sie aus Feindseligkeit ihm zu schaden trachtet; und von der Art sind die Gemüter der Zornigen. Sie schändet sich auch, wenn sie sich von der Lust oder vom Schmerze besiegen läßt; ferner, wenn sie sich verstellt und in ihren Handlungen und Reden heuchelt und lügt; endlich, wenn sie bei ihren Handlungen und Bestrebungen kein Ziel verfolgt, sondern unbesonnen ihr Tun dem Zufall überläßt, während die Pflicht gebietet, selbst die unbedeutendsten Dinge auf einen Zweck zu beziehen. Zweck vernünftiger Wesen aber ist, die vernunftgemäßen Gesetze des Staates von der allerältesten VerfassungDas Weltall wird mit einem großen Staate verglichen, der durch ein einheitliches Gesetz, das für alle Menschen dieselbe Gültigkeit hat, regiert wird. zu befolgen.

\abschnitt

Die Dauer des menschlichen Lebens ist ein Augenblick, das Wesen ein beständiger Strom,Nach der Ansicht der alten Philosophen verändert sich in der Körperwelt alles jeden Augenblick. Darum sagte Heraklit (ein griechischer Philosoph um 500 v. Chr.): Man kann nicht zweimal in denselben Strom steigen. die Empfindung eine dunkle Erscheinung, der Leib eine verwesliche Masse, die Seele ein Kreisel, das Schicksal ein Rätsel, der Ruf etwas Unentschiedenes. Kurz, was den Körper betrifft, ist ein schneller Fluß, was die Seele angeht, Träume und Dunst, das Leben ist ein Krieg, eine Haltestelle für Reisende, der Nachruhm ist Vergessenheit. Was kann uns da sicher leiten? Nur eins: die Philosophie. Und ein Philosoph sein heißt: den Genius in uns vor jeder Schmach, vor jedem Schaden bewahren, die Lust und den Schmerz besiegen, nichts dem Zufall überlassen, nie zur Lüge und Verstellung greifen, fremden Tun und Lassens unbedürftig sein, alle Begegnisse und Schicksale als von daher kommend aufnehmen, von wo wir selbst ausgegangen sind, endlich den Tod mit Herzensfrieden erwarten und darin nichts anderes sehen als die Auflösung in die Urstoffe, woraus jedes Wesen zusammengesetzt ist. Wenn aber für die Urstoffe selbst darin nichts Schreckliches liegt, daß jeder von ihnen beständig in einen andern umgewandelt wird, warum sollte man die Umwandlung und Auflösung aller Dinge mit betrübtem Auge ansehen? Das ist ja der Natur gemäß, und was mit der Natur übereinstimmt, ist kein Übel.

\end{document}
