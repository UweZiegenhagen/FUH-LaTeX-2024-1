%!TEX TS-program = Arara
% arara: pdflatex: {shell: yes}
\documentclass[12pt,ngerman,aspectratio=169]{beamer}

\usepackage{booktabs}
\usepackage{babel}
\usepackage{graphicx}
\usepackage{csquotes}
\usepackage{xcolor}
\usepackage{blindtext}
\usetheme{Metropolis}


\author{Uwe Ziegenhagen}
\title{Meine erste Präsentation}

\begin{document}

\begin{frame}
\maketitle
\end{frame}

\begin{frame}
\tableofcontents
\end{frame}


\section{Einleitung}

\begin{frame}
\frametitle{Hallo Welt}

\begin{itemize}
\item 1
\item 2
\item 3345
\item 53453
\item 345345
\item 34534
\end{itemize}
\end{frame}


\section{Analyse}

\begin{frame}
\frametitle{Hallo Welt}

\begin{itemize}
\item 1
\item 2
\item 3345
\item 53453
\item 345345
\item 34534
\end{itemize}
\end{frame}



\section{Fazit}

\begin{frame}
\frametitle{Hallo Welt}

\begin{itemize}
\item 1
\item 2
\item 3345
\item 53453
\item 345345
\item 34534
\end{itemize}
\end{frame}

\begin{frame}
\frametitle{fsfs}
\framesubtitle{Hallo Fernuni Hagen}



\begin{equation}
-\frac{p}{2} \pm \sqrt{  \left(\frac{p}{2}\right)^2 - q  }
\end{equation}

\end{frame}

\begin{frame}
\frametitle{Mauze}

\begin{center}
\includegraphics[width=0.5\textwidth]{../Bilder/Katze}
\end{center}

\end{frame}

\begin{frame}
\frametitle{Mauze}

\begin{columns}
\begin{column}{0.495\textwidth}
Sage zu dir in der Morgenstunde: Heute werde ich mit einem unbedachtsamen, undankbaren, unverschämten, betrügerischen, neidischen, ungeselligen Menschen zusammentreffen. 
\end{column}
\begin{column}{0.495\textwidth}
Sage zu dir in der Morgenstunde: Heute werde ich mit einem unbedachtsamen, undankbaren, unverschämten, betrügerischen, neidischen, ungeselligen Menschen zusammentreffen. 
\end{column}

\end{columns}
\end{frame}



\end{document}