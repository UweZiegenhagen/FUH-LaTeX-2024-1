%!TEX TS-program = Arara
% arara: pdflatex: {shell: yes}
% arara: biber
% arara: pdflatex: {shell: yes}

\documentclass[12pt,ngerman,parskip=half]{scrartcl}

\usepackage{babel}
\usepackage{blindtext}

\usepackage[style=authortitle-icomp,backend=biber]{biblatex}
\usepackage[babel,german=quotes]{csquotes}

\addbibresource{Literatur.bib}

\usepackage{hyperref}
\hypersetup{
    bookmarks=true,                     % show bookmarks bar
    unicode=false,                      % non - Latin characters in Acrobat’s bookmarks
    pdftoolbar=true,                        % show Acrobat’s toolbar
    pdfmenubar=true,                        % show Acrobat’s menu
    pdffitwindow=false,                 % window fit to page when opened
    pdfstartview={FitH},                    % fits the width of the page to the window
    pdftitle={My title},                        % title
    pdfauthor={Author},                 % author
    pdfsubject={Subject},                   % subject of the document
    pdfcreator={Creator},                   % creator of the document
    pdfproducer={Producer},             % producer of the document
    pdfkeywords={keyword1, key2, key3},   % list of keywords
    pdfnewwindow=true,                  % links in new window
    colorlinks=true,                        % false: boxed links; true: colored links
    linkcolor=blue,                          % color of internal links
    filecolor=blue,                     % color of file links
    citecolor=blue,                     % color of file links
    urlcolor=blue                        % color of external links
}
\begin{document}

\blindtext

\cite{Duck2000}

Was ich auch immer sein mag, es ist doch nur ein wenig Fleisch, ein schwacher Lebenshauch und die leitende Vernunft. Laß die Bücher,Antonin war durch seine Regentenpflichten so sehr beschäftigt, daß er seine Neigung zum Lesen unterdrücken mußte. (Bücher konnten damals nur die Reichen kaufen, die philosophischen Schriften wurden oft mit mehreren Talenten, d. h. mehreren Tausend Talern bezahlt.) die Zerstreuung, es fehlt dir die Zeit. Betrachte dich als einen, der im Begriff ist zu sterben, verachte dieses Fleisch: Blut, Knochen, ein zerbrechliches Gewebe, aus Nerven, Puls- und Blutadern zusammengeflochten. Betrachte diesen Lebenshauch selbst; was ist er? Nur Wind, und nicht einmal immer derselbe, sondern jeden Augenblick ausgeatmet und wieder eingeatmet. Das Dritte ist die gebietende Vernunft. Auf folgendes mußt du bedacht sein: Du bist alt; gib nicht mehr zu, daß sie eine Sklavin sei, daß sie durch einen wilden Trieb dahingerissen werde oder gegen das jetzige Geschick murre oder durch das künftige erschüttert werde.\footnote{\blindtext}

A \cite{Chen2018}

\clearpage

B \parencite{Chen2018}

C \blindtext\footcite{Chen2018} 

D \footcite{Duck2000}

E \blindtext\footcite{Chen2018}

F \citeauthor{Knuth1984} hat in seinem Werk  \citetitle{Knuth1984}, das er im Jahr  \citeyear{Knuth1984} veröffentlicht hat, wichtige Aussagen getroffen.

\printbibliography[title={Artikel},type=article]

\printbibliography[title={Bücher},type=book]


\end{document}